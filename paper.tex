\documentclass[prb,preprint]{revtex4-1} 
% The line above defines the type of LaTeX document.
% Note that AJP uses the same style as Phys. Rev. B (prb).

% The % character begins a comment, which continues to the end of the line.

\usepackage{amsmath}  % needed for \tfrac, \bmatrix, etc.
\usepackage{amsfonts} % needed for bold Greek, Fraktur, and blackboard bold
\usepackage{graphicx} % needed for figures

\begin{document}

% Be sure to use the \title, \author, \affiliation, and \abstract macros
% to format your title page.  Don't use lower-level macros to  manually
% adjust the fonts and centering.

\title{Outline of Kinesthetic Kinematics }
% In a long title you can use \\ to force a line break at a certain location.

\author{Thomas S. Allen}
\email{thallen@pdx.edu} % optional
\altaffiliation[permanent address: ]{101 Main Street, 
  Anytown, USA} % optional second address
% If there were a second author at the same address, we would put another 
% \author{} statement here.  Don't combine multiple authors in a single
% \author statement.
\affiliation{Department of Physics, Portland State University, Portland, OR 97207-0751}
% Please provide a full mailing address here.


% See the REVTeX documentation for more examples of author and affiliation lists.

\date{\today}

\begin{abstract}

\end{abstract}
% AJP requires an abstract for all regular article submissions.
% Abstracts are optional for submissions to the "Notes and Discussions" section.

\maketitle % title page is now complete


\section{Introduction} % Section titles are automatically converted to all-caps.
% Section numbering is automatic.

Physics concepts are particularly suited to be explored via kinesthetic activities.  Indeed, concepts in physics, such as kinematics and dynamics are central to our kinesthetic experience of the world.  A number of authors have begun incorporating kinesthetic activities 

\subsection{Active learning} 
Cite Holmes \& Weiman 2018

\subsection{Active learning activities in literature} 

\subsection{Kinesthetic Activities}

\subsubsection{Difficulties implementing kinesthetic activities}



\subsection{The Technology}

\section{The Activity}

\subsection{1-D}

\subsubsection{Constant velocity}

Comparing x(t) slopes with measured velocity

Out and Back - same area under v(t) graph

Out and Back - different areas under v(t) graph

\subsubsection{Constant acceleration}

Examining x(t) graphs with constant {\it a}

\subsubsection{Rotational Motion}


\subsection{3-D}

Constant velocity in one direction, step-wise changing velocity in other

Determining x-y position plots from x(t) and y(t) plots

\subsubsection{Rotational Motion}


\subsection{Novel approaches to 

\section{Conclusion}

Encourage students to think of their own ways to do the activities.

Teacher rolling ball - mention Murdock workshop

Inquiry based approach to activities


%\begin{figure}[h!]
%\centering
%\includegraphics{GasBulbData.eps}
%\caption{Pressure as a function of temperature for a fixed volume of air.  
%The three data sets are for three different amounts of air in the container. 
%For an ideal gas, the pressure would go to zero at $-273^\circ$C.  (Notice
%that this is a vector graphic, so it can be viewed at any scale without
%seeing pixels.)}
%\label{gasbulbdata}
%\end{figure}

%\begin{figure}[h!]
%\centering
%\includegraphics[width=5in]{ThreeSunsets.jpg}
%\caption{Three overlaid sequences of photos of the setting sun, taken
%near the December solstice (left), September equinox (center), and
%June solstice (right), all from the same location at 41$^\circ$ north
%latitude. The time interval between images in each sequence is approximately
%four minutes.}
%\label{sunsets}
%\end{figure}




\section{Endnotes and references}



\begin{acknowledgments}


\end{acknowledgments}


\begin{thebibliography}{99}
% The numeral (here 99) in curly braces is nominally the number of entries in
% the bibliography. It's supposed to affect the amount of space around the
% numerical labels, so only the number of digits should matter--and even that
% seems to make no discernible difference

\end{thebibliography}



\end{document}
